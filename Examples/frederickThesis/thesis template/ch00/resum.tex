Les plataformes d'intercanvi de recursos multimèdia a Internet contenen grans quantitats de contingut creat pels seus usuaris.
Habitualment, aquest contingut no està ben anotat, i això fa que la seva indexació no sigui una tasca fàcil. Aconseguir que aquest contingut sigui accessible pels altres usuaris suposa un repte important, el qual no és només específic d'aquest tipus de plataformes. En general, l'anotació de contingut és un problema comú en molts tipus de sistemes d'informació.
En aquesta tesi, ens focalitzem en aquest problema i proposem mètodes per ajudar els usuaris a anotar, d'una manera més completa i uniforme, el contingut creat per ells mateixos. Concretament, treballem amb sistemes d'etiquetatge -- \emph{tagging} -- i proposem mètodes per recomanar etiquetes -- \emph{tags} -- durant el procés d'anotació del contingut.
Per aconseguir això, analitzem la manera com els altres continguts de la plataforma d'intercanvi han estat etiquetats prèviament. Aquesta informació s'anomena \emph{folksonomia}.
Avaluem la tasca de recomanar tags utilitzant diverses metodologies, amb o sense la participació d'usuaris, i en el context de sistemes de tagging a gran escala.
Particularment, ens focalitzem en el cas de la recomanació de tags en plataformes d'intercanvi de sons i, a part de testar el funcionament de diferents mètodes en aquest escenari, també analitzem l'impacte d'un d'aquests mètodes en el sistema de tagging d'una plataforma d'intercanvi de sons real.
De fet, de resultes d'aquesta tesi, centenars d'usuaris fan servir diàriament un dels sistemes proposats de recomanació de tags en aquesta plataforma d'intercanvi.
A més a més, també explorem un nou enfocament per als sistemes de recomanació de tags que, a part de nodrir-se de la informació de la folksonomia, incorpora una ontologia amb informació sobre l'àmbit del so que serveix per guiar els usuaris durant el procés d'anotació de contingut.
En general, aquesta tesi contribueix a l'avenç de l'estat de l'art dels sistemes de tagging i de recomanació de tags basats en folksonomies, i explora direccions interessants per continuar investigant.
Tot i que la nostra recerca està motivada pels reptes particulars que proposen les plataformes d'intercanvi de sons i està avaluada principalment en aquest context, creiem que les metodologies que proposem poden ser generalitzades fàcilment i utilitzades en altres plataformes d'intercanvi.