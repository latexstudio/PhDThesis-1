%!TEX root = ../thesis_a4.tex

\chapter{Background}
\label{chap:background}

\section{Introduction}

\section{Music background}
\label{sec:music_background}

\subsection{Indian art music}
\subsection{Melody in Indian art music}

\subsubsection{Tonic in Indian art music}
\subsubsection{Svaras}
\subsubsection{Aroh-Avroh}
\subsubsection{Gamakas and Ornaments}
\subsubsection{Raga Concept}
\subsubsection{Characteristic Melodic Phrases}
\subsubsection{Chalan}

\subsubsection{Nyas}
\label{sec:backgroung_nyas_description}

Dey presents various interpretations and perspectives on the concept of ny\={a}s in Hindustani music according to ancient, medieval and modern authors~\cite{Dey2008}. In the context of its current form, the author describes ny\={a}s as that process in a performance of a r\={a}g where an artist pauses on a particular svar\footnote{The seven solf\`{e}ge symbols used in Indian art music are termed as svars. It is analogous to note in western music but conceptually different.}, in order to build and subsequently sustain the format of a r\={a}g, the melodic framework in Indian art music~\cite[p. 70]{Dey2008}\cite{KKG_SS13}. Dey elaborates the concept of ny\={a}s in terms of action, subject, medium, purpose and effect associated with it. Typically, occurrence of a ny\={a}s delimits melodic phrases (motifs), which constitute one of the most important characteristic of a r\={a}g. Analysis of ny\={a}s is thus a crucial step towards melodic analysis of Hindustani music. In particular, automatically detecting occurrences of ny\={a}s (from now on referred as ny\={a}s segments) will aid in computational analyses such as melody segmentation, motif discovery, r\={a}g recognition and music transcription~\cite{GopalJNMR2012, Rao2014}. However, detection of ny\={a}s segments is a challenging computational task, as the prescriptive definition of ny\={a}s is very broad, and there are no fixed set of explicit rules to quantify this concept~\cite[p. 73]{Dey2008}. It is through rigorous practice that a seasoned artist acquires perfection in the usage of ny\={a}s, complying to the r\={a}g grammar and exploring creativity through improvisation at the same time. 


\subsubsection{Sections}

\section{Relevant Work in Melodic Analyses}
\subsection{Tonic Identification}
\subsection{Key and Mode Identification}
\subsection{Predominant melody estimation}
\subsection{Melody Representation, Segmentation, Similarity, Search and Discovery}
\subsubsection{For Indian Art Music}
\subsubsection{For Other Music Traditions}
\subsection{R\={a}ga recognition}
\subsection{Corpus level melodic analysis}

SOME MATERIAL FOR THIS CHAPTER (IN NO ORDER)

In computational analysis of \gls{iam}, \gls{nyas} segment detection has not received much attention in the past. To the best of our knowledge, only one study with the final goal of spotting melodic motifs has indirectly dealt with this task~\citep{Ross2012}. In it, the authors considered performances of a single r\={a}g and focused on a very specific \gls{nyas} \gls{svara}, corresponding to a single scale degree: the fifth with respect to the tonic, the `Pa' \gls{svara}. This \gls{svara} is considered as one of the most stable \glspl{svara}, and has minimal pitch deviations. Thus, focusing on it oversimplified the methodology developed in~\cite{Ross2012} for \gls{nyas} segment detection. \TODO{Should we move this to state of the art? + svara notation should be replaced by a glossary item?} 

A related topic is the detection of specific \glspl{alankar} and characteristic phrases (also referred as {\it Pakads}) in melodies in Indian art music~\cite{Datta2007, Pratyush2010, Ross2012b, Ishwar2013}. These approaches typically exploit pattern recognition techniques and a set of pre-defined melodic templates. A nearest neighbors classifier with a similarity measure based on dynamic time warping (DTW) is a common method to detect patterns in melodic sequences~\cite{Pratyush2010, Ross2012b}. In addition, it is also the most accurate~\cite{Xi06ICML} and extensively used approach for time series classification in general (cf.~\cite{Wang12DMKD}). Notice that the concept of landmark has been used elsewhere, with related but different notions and purposes. That is the case with time series similarity~\cite{Perng00ICDE}, speech recognition~\cite{Jansen08JASA,Chen12ICASSP}, or audio identification~\cite{Duong13ICASSP}.


%
%\section{Motif discovery in time-series}
%\subsection{Lower-bounding techniques for DTW}

\section{{Relevant Scientific Concepts}}
\subsection{Dynamic Time Warping}
\subsection{Lower Bounds on DTW}
\subsection{Term Frequency Inverse Document Frequency?}



