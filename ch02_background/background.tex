%!TEX root = ../thesis_a4.tex

\chapter{Background}
\label{chap:background}

\section{Introduction}

In this chapter we present an overview of the terminology used in this work, music and scientific background, and literature review relevant to this thesis. In order to avoid possibilities of misunderstanding and confusions arising because of s

we provide a brief discussion on the terminology used in this thesis, present an overview of the music and scientific background needed to better understand the concepts and present a review of the existing research work relevant to the topics addressed in this thesis. 



\section{Terminology}
\label{sec:terminology}

The description of the terms that we provide here is not intended to be the formal definition. Since most of the terms 

Since the thesis focuses on the melodic description of \gls{iam}, it is important to clearly understand the meaning of melody in the context of this thesis. Our aim here is not to formally define melody in an universal sense, which as we see from the literature has been a challenge in itself, with almost every definition falling short of considering some aspect or the other XXXX. It would not be an overstatement to say that there is no agreed definition of melody that suits every context. For a review of interesting definitions of melody we refer to XXXX. Before we present the definition of melody that we consider in this work, it is important to understand the performance or concert setup in \gls{iam}. Since this music tradition is performance-based, even the studio recordings follow the same setup. Performances in \gls{iam} have a clearly defined concept of a lead artist, who plays the central role (also literally positioned in the center of the stage) and all other instruments are considered the accompaniments. In this context, combining the definitions given by~\cite{paiva2006melody,kim2000analysis,levitin2002memory} with a small modifications covers to a large extent the scope of melody in \gls{iam}. Their definitions in the respective order are: \textquote{``\textit{the dominant individual pitched line in a musical ensemble}''} and \textquote{``\textit{an auditory object that emerges from a series of transformations along the six dimensions: pitch, tempo, timbre, loudness, spatial location, and reverberant environment}''}.  The first definition falls short of considering several other dimensions of sound that are important in the perception or production of melody, which are present in the second definition. The concept of audio as a mixture of sounds from multiple instruments (`\textit{ensemble}') is missing in the second definition, which the first definition takes into account. The idea of continuity and smoothness of melody is expressed by `\textit{line}' in the former and by `\textit{series of transformations}' in the latter. If we combine these two definitions we can rewrite the definition as \textquote{``\textit{an auditory object that emerges from the component of the ensemble sound corresponding to the lead artist, through continuous series of transformations along the six dimensions: pitch, tempo, timbre, loudness, spatial location, and reverberant environment}''}. However, we only consider the pitch dimension of melody in this thesis. Thus, loosely speaking, for all practical purposes within the scope of this thesis, we represent melody by continuous pitch corresponding to the lead artist in the audio recording, also referred to as predominant pitch. This definition does not explicitly take into account the rare case of two lead artists. Since in a majority of such cases the two vocal lines are active sequentially in time, our definition encompasses this scenario. 

and regard `\textit{dominant}' as the lead voice (making it specific to \gls{iam}) 


Melody
Melodic Pattern
Melodic Fragment or Segment
Melodic phrase
Melodic motif
Recurrence, melodic recurrence, repetition
Raga motif

\section{Music Background}
\label{sec:music_background}

\subsection{Indian Art Music}

\subsection{Melodies in Indian Art Music}

\subsubsection{Tonic in Indian art music}

\subsubsection{Svaras}

\subsubsection{Aroh-Avroh}

\subsubsection{Gamakas and Ornaments}

\subsubsection{Raga in Indian Art Music}

\paragraph{Phase-based and Scale-based ragas}
\paragraph{Allied ragas}

\subsubsection{Characteristic Melodic Phrases}

\subsubsection{Chalan}

\subsubsection{Nyas}
\label{sec:backgroung_nyas_description}

Dey presents various interpretations and perspectives on the concept of ny\={a}s in Hindustani music according to ancient, medieval and modern authors~\cite{Dey2008}. In the context of its current form, the author describes ny\={a}s as that process in a performance of a r\={a}g where an artist pauses on a particular svar\footnote{The seven solf\`{e}ge symbols used in Indian art music are termed as svars. It is analogous to note in western music but conceptually different.}, in order to build and subsequently sustain the format of a r\={a}g, the melodic framework in Indian art music~\cite[p. 70]{Dey2008}\cite{KKG_SS13}. Dey elaborates the concept of ny\={a}s in terms of action, subject, medium, purpose and effect associated with it. Typically, occurrence of a ny\={a}s delimits melodic phrases (motifs), which constitute one of the most important characteristic of a r\={a}g. Analysis of ny\={a}s is thus a crucial step towards melodic analysis of Hindustani music. In particular, automatically detecting occurrences of ny\={a}s (from now on referred as ny\={a}s segments) will aid in computational analyses such as melody segmentation, motif discovery, r\={a}g recognition and music transcription~\cite{GopalJNMR2012, Rao2014}. However, detection of ny\={a}s segments is a challenging computational task, as the prescriptive definition of ny\={a}s is very broad, and there are no fixed set of explicit rules to quantify this concept~\cite[p. 73]{Dey2008}. It is through rigorous practice that a seasoned artist acquires perfection in the usage of ny\={a}s, complying to the r\={a}g grammar and exploring creativity through improvisation at the same time. 

\subsubsection{Recurring Melodic Patterns in IAM}

\section{Relevant Work in Indian Art Music}

\subsection{Tonic Identification}

\subsection{R\={a}ga Recognition}

\subsection{Melodic Pattern Processing}



\section{Relevant Work (in MIR??) or (Other Music Tradition?)}

\subsection{Key and Mode Recognition}

\subsection{Pattern Processing in Music}

% Patterns at different time scales: 1) Sections, 2) Motifs, 3) Stanzas 4) Chord sequences? 

% Focus on Motifs/Riffs 

% Subparts: Melody representation, Melody segmentation, Melodic similarity, Discovery methodology, Redundancy reduction etc

\subsection{Corpus level melodic analysis}

%SOME MATERIAL FOR THIS CHAPTER (IN NO ORDER)

%In computational analysis of \gls{iam}, \gls{nyas} segment detection has not received much attention in the past. To the best of our knowledge, only one study with the final goal of spotting melodic motifs has indirectly dealt with this task~\citep{Ross2012}. In it, the authors considered performances of a single r\={a}g and focused on a very specific \gls{nyas} \gls{svara}, corresponding to a single scale degree: the fifth with respect to the tonic, the `Pa' \gls{svara}. This \gls{svara} is considered as one of the most stable \glspl{svara}, and has minimal pitch deviations. Thus, focusing on it oversimplified the methodology developed in~\cite{Ross2012} for \gls{nyas} segment detection. \TODO{Should we move this to state of the art? + svara notation should be replaced by a glossary item?} 
%
%A related topic is the detection of specific \glspl{alankar} and characteristic phrases (also referred as {\it Pakads}) in melodies in Indian art music~\cite{Datta2007, Pratyush2010, Ross2012b, Ishwar2013}. These approaches typically exploit pattern recognition techniques and a set of pre-defined melodic templates. A nearest neighbors classifier with a similarity measure based on dynamic time warping (DTW) is a common method to detect patterns in melodic sequences~\cite{Pratyush2010, Ross2012b}. In addition, it is also the most accurate~\cite{Xi06ICML} and extensively used approach for time series classification in general (cf.~\cite{Wang12DMKD}). Notice that the concept of landmark has been used elsewhere, with related but different notions and purposes. That is the case with time series similarity~\cite{Perng00ICDE}, speech recognition~\cite{Jansen08JASA,Chen12ICASSP}, or audio identification~\cite{Duong13ICASSP}.


%
%\section{Motif discovery in time-series}
%\subsection{Lower-bounding techniques for DTW}
\section{Scientific Background}

\subsection{Distance Measures}
\subsubsection{Euclidean Distance}
\subsubsection{Dynamic Time Warping}
\subsubsection{KL Divergence}
\subsubsection{Bhattacharya}

\subsection{Indexing of Time-series}
\subsubsection{Lower Bounds for DTW distance}

\subsection{Evaluation Measures for Information Retrieval}
\subsubsection{Precision and Recall}
\subsubsection{Mean Average Precision}
\subsubsection{Expert Evaluation?}

\subsection{Machine Learning Concepts}
\subsubsection{Classification Methods}
\subsubsection{Evaluation Strategies}

\subsection{Complex Networks}


\subsection{Statistical Testing}
\subsubsection{Mann-Whitney U Test}
\subsubsection{Wilcoxin Test}
\subsubsection{Signed Rank Test}
\subsubsection{McNemar's Test}
\subsubsection{Holm-Bonferroni Method}



