%!TEX root = ../thesis_a4.tex

\chapter{Music corpus and datasets}\label{chap:datasets}

\section{Introduction}
\label{sec:corpus_intro}

Data is a fundamental requirement in the development of information retrieval technologies. A research corpora is a collection of data compiled to study a specific research problem. A well designed research corpora is representative of the domain under study, and is used for the development and evaluation of approaches (???). In \gls{mir} a considerable number of the computational approaches are data-driven, and hence, a well curated research corpora becomes a key factor in determining the success of these approaches. Due to the importance of a good representative corpora in research in general, the criterion for designing a reserch corpora becomes a fundamental task in itself.

 


\begin{itemize}
	\item Data is needed for dev of MIR approaches
	\item MIR is primarily a data driven field
	\item How should be the research corpus used for these approaches
	\item What are the components of these corpus, how its used, in which modes
	\item How does it develop
	\item How is it curated/selected/ what should be the criterions. 
	\item Issue of reproducibility in research, sharable corpus, open access corpus
	\item examples of researhc corporas 
	\item openly available corporas
	
\end{itemize}


\section{CompMusic corpora}

\subsection{Creating the corpora}

\subsection{Carnatic music corpus}

\subsection{Hindustani music corpus}

\section{Datasets}

\subsection{Tonic dataset}

\subsection{Nyas dataset}

\subsection{Melodic similarity dataset}

\subsection{Raga recognition dataset}

\subsection{Saraga dataset}








