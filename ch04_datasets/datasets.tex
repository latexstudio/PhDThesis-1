%!TEX root = ../thesis_a4.tex

\chapter{Music corpus and datasets}\label{chap:datasets}

\section{Introduction}
\label{sec:corpus_intro}

Data is a fundamental requirement in the development of information retrieval technologies~\citep{manning2008introduction}. A research corpora is a collection of data compiled to study a specific research problem. A well designed research corpora is representative of the domain under study used for the development and evaluation of approaches. It is practically infeasible to work with the entire universe of data. Therefore, to ensure the scalability of information retrieval technologies to real-world scenarios, it is to important to develop and evaluate the approaches using representative data corpus. In addition to catering the scalability of the approaches, an easily (or, better publicly) accessible data corpora provides a common ground for researchers to evaluate their methods, and thus, accelerate the advancement of the knowledge. 

Not every computational task would require an entire reseach corpus for development and evaluation of approaches. Typically a subset of the corpus is used in a specific research task. We call this subset a test corpus or test dataset. Test dataset is a static collection of data specific to an experiment, as opposed to a research corpus, which is an evolving collection of data. Therefore, different versions of the test dataset used in a specific experiment should be retained for ensuring the reproducibility of the results. The models build over test dataset can later be extended to entire research corpus. Note that a test dataset should not be confused with the training and testing split of a dataset often performing in a cross validation experimental setup.


In \gls{mir} a considerable number of the computational approaches follow a data-driven methodology, and hence, a well curated research corpora becomes a key factor in determining the success of these approaches. Due to the importance of a good corpora in research, building a corpora in itself becomes a fundamental research task. It has been studied in many fields such as XXX, XXX and XXX. \Gls{mir} is relatively a new research area within information retrieval, which has primarily gained popularity in last 2 decades. Even today a majority of the studies in \gls{mir} use an ad-hoc procedures to build a collection of data to be used for the experiments. Quite often the audio data comes from the researcher's personal music collection. Availability of a good representative research corpora has been a challenge in \gls{mir}. This to an extend can be attributed to the large variety of research problems studied within \gls{mir}, lack of standardized methodologies for data collection and annotation, and most of all, the constrained posed by the copyrighted content.

Recently there have been some efforts to compile large collections of music related data to build a research corpora that can be used to study a number of computational tasks in \gls{mir}. One such example is the Million Song Dataset (MSD)\TODO{ref}, which has been used in a number of studies for a variety of tasks\TODO{ref}. However, owing to the copyright issues, the audio for the recordings in MSD is not available. Building a good representative research corpora, which in the field of \gls{mir} would typically mean compilation of a large number of music recordings and it's related metadata is a big effort. A successful sustainable strategy could be to make it a community effort. An endeavor in this direction is AcousticBrainz\TODO{ref}, which aims to crowd source acoustic information for all music in the world and make it available to the public. Data in Acousticbrainz is indexed by unique MusicBrainz identifiers. MusicBrainz\TODO{ref} is an open music encyclopedia that collects music metadata and makes it available to the public. Such open repositories can also serve as corpora for a variety of research problems in \gls{mir}. 

\TODO{Mention COFLA somewhere}

While there is a growing effort towards building and using a representative sizable corpora in research, there are not many studies that address formally the task of building a good research corpora. There is a lack of studies that discuss the criterion for determining the goodness of a corpora for a particular task and systematic ways to compile and curate research corpora. Some recent efforts towards this direction includes the work by XXX in which the author present unified ways to describe annotated \Gls{mir} datasets. There have been some efforts to define a specifications to store annotations in a more unified way to promote reproducibility and easy access of the corpora. As a part of CompMusic project in XXX presents a set of design criterion for building research corpora that is representative of a given domain of study. These criterions are based around considerations such as purpose, coverage, completeness, quality and reusability. 

%The corpus developed in the CompMusic project for studying a number of problems in \Gls{mir} of \Gls{iam} is based on these criterion.

In this chapter we describe the CompMusic research corpora built for studying a number of computational tasks in \gls{mir} of \gls{iam}. Along with the description of the corpora we also briefly discuss the methodology  and the design criterion used to compile and curate the corpora. Note that the sources used for compiling the corpora are not comprehensive. Our primary aim is to present the approach we used to build the corpus than to justify a specific data source. In addition to the CompMusic corpora we also briefly describe the individual test datasets (Section XX) built for studying specific aspect of melody in \gls{iam}. These test datasets are used for evaluating different approaches described in subsequent chapters. 

%
%\begin{itemize}
%	\item Data is needed for dev of MIR approaches
%	\item MIR is primarily a data driven field
%	\item How should be the research corpus used for these approaches
%	\item What are the components of these corpus, how its used, in which modes
%	\item How does it develop
%	\item How is it curated/selected/ what should be the criterions. 
%	\item Issue of reproducibility in research, sharable corpus, open access corpus
%	\item examples of researhc corporas 
%	\item openly available corporas
%	\item ways to access them
%\end{itemize}


\section{CompMusic corpora}

As described earlier in Section XXX, in CompMusic project we work on data-driven computational approaches to describe music recordings and emphasize the use of domain knowledge of a particular music tradition. This project focuses on five different music traditions: Arab-Andalusian (Maghreb), Beijing Opera (China), Turkish makam music (Turkey), Hindustani (North-India) and Carnatic (South-India) music. One of the key ideas in CompMusic project is that there are some universal music concepts such as melody and rhythm common across different music traditions, but many important aspects of a music recording can be better understood and appreciated by focusing on the specificities of the music tradition. Therefore, a significant effort in this project has been to compile a representative research corpora that captures the specificities of different music traditions considered in the project. In addition, an effort is made to define the design criterion that can be used to build good research corporas. In the subsequent section we enumerate the criterion chosen to build the CompMusic corpora.

\subsection{Criterion for Creating Corpora}
\label{sec:corpus_criterion_for_corpora}

\cite{serra:14:corpus} enumerates a set of criterion for building a good representative research corpora that have been used in the CompMusic project. These criterion are listed below along with a brief description.

\subsubsection{Purpose}

The purpose for building a corpora should be clearly specified, which includes defining the research problem that needs to be addressed and the research approach that will be used. In CompMusic we want to develop methodologies to extract musically meaningful features from audio music recordings, primarily related to melody and rhythm. The approaches taken are based on signal processing and machine learning techniques. A research corpus should take these factors into account.

\subsubsection{Coverage}

As mentioned above, a good corpus should be representative of the domain under study. Coverage is a measure for the representativeness of a corpus with respect to the concepts to be studied. Given the quantitative approach of the CompMusic project, we need sufficient instances of each concept for the data to be statistically significant. For melodic analsis, we need to have audio recordings and appropriate accompanying information that represent the diversities present in the melodic aspects of Hindustani and Carnatic music such as different forms, artists from different schools of music, and the variety of \glspl{raga} frequently performed in each music culture. 

\subsubsection{Completeness}

To successfully use data in a meaningful analyses it should be complemented by the appropriate metadata. Completeness indicates the completeness of the metadata for each audio recording. For the CompMusic corpus it mainly refers to the completeness of the editorial metadata and of the descriptive information accompany each audio recording. 

\subsubsection{Quality}

The quality of the data in a corpus should be good. In our case it means that the audio has to be well recorded and the accompanying metadata should be accurate. We use commercial produced well recorded audio data and the accompanying information is obtained from reliable sources. Quite often the information collected from reliable sources such as editorial metadata on the CD cover is erroneous, in such cases the metadat is validated by experts. 

\TODO{Write a paragraph on the quality and the storing format of the audio collection}

\subsubsection{Reusability}

Reusability of the corpus is fundamental for reproducibility of the research results. An important aspect that impacts reusability is the ease of access to the corpus. This implies that the corpus should be available for the community to access and it has to be well structured for an easy integration. We address reusability by emphasizing the use of open repositories that are either already suitable or can be easily adapted to our needs. For organizing editorial metadata we use MusicBrainz. In addition, we make efforts for an each access of the corpus through the Dunya API.

\TODO{Explain the resuability component of the corpora, explain dunya API and ways to access data etc. Also emphasize the sharing and public availability of the datasets}




%\begin{itemize}
%\item the kind of research done in compmusic
%\item important of a good corpora in this project
%\item design principles followed for building this corpora
%\item Issue of open access, that makes it more powerful
%\item Structured way to access the corpora, through APIs
%\item Complemented by metadata for overall approaches and incorporating domain knowledge
%\item describing specific Hindustani and carnatic Datasets along with thir attributes and measurement of different parameters used to measure a good corpora.
%\end{itemize}


We now proceed to describe the specificities of the Carnatic and the Hindustani music corpus, both of which are a part of the CompMusic corpus. We will describe the data that constitute the corpus, the unit of the data sample, references for measuring completeness and coverage of the data, resources for obtaining complementary information about musical concepts and XXX. Compilation of both the corpora as summarized in the subsequent sections is a collective effort by CompMusic team. The description below is also based primarily on our earlier work presented by~\cite{CM_Corpora_Ajay14}.


\subsection{Carnatic music corpus}
\label{sec:corpus_carnatic_music_corpus}

\Gls{raga} is the melodic and \gls{tala} is the rhythmic frameworks in both Carnatic and Hindustani music~(Section~XXX). They are the two key musical concepts in \gls{iam} around which the entire music is composed, performed, organized and taught. As a result of which these concepts have been the main considerations based on which both the corpora are compiled. Both the corpora primarily comprise audio recordings and its associated editorial metadata and are the ones used by signal processing and machine learning approaches. In addition to that there are lyrics, scores, contextual information on music concepts and community (social) information from online music forums and other sources.

There are several music tradition specific considerations taken into account while building the corpus. A concert, also referred as a (Kach\={e}ri), is the natural unit of Carnatic music. It is the unit typically considered in organization and digital distribution of Carnatic music content. Though Carnatic music is improvisatory in nature, it is predominantly based on compositions. Most of the compositions are to be sung, as a result of which, vocal music is dominant in Carnatic music. Even in instrumental music, the lead artist aims to mimic the vocal singing~\citep{Viswanathan2004}. Based on these considerations, we consulted expert musicians and musicologists, such as T.~M.~Krishna to arrive at a representative audio collection of Carnatic music.

The main institutional reference for gathering the Carnatic music collection is the \gls{mma}. The \Gls{mma} was conceived to be the institution that would set the standard for Carnatic music. Since 1929, the \Gls{mma} hosts annual conferences on music, which has eventually lead to the December music festival of Madras, one of the largest cultural events of the world. The \gls{mma} has been driving the scholarly research in Carnatic music since a long time and has influenced the evolution of the musical concepts being used. The \gls{mma} has an expert pannel that sets the standards and fix procedures for selecting artists for the music festival. Since a long time the \gls{mma} has been recording Carnatic music performances and its archive is considered as one of the main references for Carnatic music. However, the archive is not openly available online. We therefore procured the audio recordings through other commercial sources while following the musical criterion used by the \gls{mma}. We started by collecting the releases of the artists who have performed in the \gls{mma} in the last five years. Subsequently, we expanded the collection to include their teachers and other musicians popular in their era. 

One of the main record labels that specializes in Carnatic music recordings is Charsur, who has been publishing high quality commercial CDs for over 15 years now. The core of Carnatic music audio collection is from their catalog of music concerts. 
  

\begin{itemize}
	\item contents of the corpus
	\item Considerations in collecting data/music recordings
	\item whom we consulted how we went ahead for 
	\item Academic references
	\item Recording label, purchasing of material
	\item storage of data
	
\end{itemize}




The main institutional reference for Carnatic music
is the Madras Music Academy (MMA) 9
, which is a premier
institution dedicated to Carnatic music and organizes
the annual music conference in Chennai, India. The annual
Carnatic music festival is one of the largest music festivals
in the world, with a significant part of the Carnatic
music community taking part in it. The MMA has been
driving scholarly research and opinion in Carnatic music.
The MMA has a panel of experts that formulates the procedure
and standards for the selection of artists for the music
festival. The MMA has been recording concerts and its
archive can be considered a standard repository of Carnatic
music. However, the archive is not openly available online.
We thus followed the musical criteria followed by the
MMA and procured the audio from commercially available
releases. Though Carnatic music is spread across South India,
the choice of MMA as an institutional reference will
have an influence on the research corpus introducing a bias
towards the music scene in Chennai.

We wished to compile concerts over several generations
of musicians. We started with the artists that have been
performing at the MMA in the last five years, and then expanded
the collections to include their teachers, and popular
musicians of their era. The record label Charsur 10 specializes
in Carnatic music and the core of our audio collection
is from their catalog of music concerts. Hence, the corpus
consists of audio from commercially available releases
from Charsur and other music labels. The corpus presently
consists of 248 releases(concerts) with 1650 audio recordings
(346 hours) spanning 1068 compositions. The number
of other relevant music entities in the corpus is described
in Table 1 (column 2). Though we focus on concerts with
vocalist leads, we also have instrumental music releases
(mainly with Veena, Violin, Flute, Saxophone, and Mridangam
in lead). The whole audio collection is commercial
and easily accessible, but is not open and distributable.
However, efforts are underway to compile a freely available
open collection of Carnatic music.
The editorial metadata associated with each release has
been stored and organized in MusicBrainz. The primary
metadata associated with each concert is the name of the
release, the lead and the accompanying artists, and the musical
instruments in the concert. For each audio recording
contained in the release, the relevant metadata are the artists
performed on the track, the name of the composition/s and
the composer, raga/s, tala/s, musical form/s. MusicBrainz
assigns a unique identifier (MBID) for each entity in MusicBrainz,
such as the artist, composer, instrument, recording,
work, and a release. This helps to organize the metadata
in an effective way. All the editorial metadata was entered
using Roman alphabet and a roman transliteration was
used when the language of the release was not English. The
raga and tala information was added as tags, though a recent
version of MusicBrainz supports work attributes with
which this information can be stored and accessed better.
Efforts are underway to convert the existing tags into work
attributes.
Since Carnatic music is predominantly a vocal music tradition,
lyrics play an important role. A significant part of
the rendition of a composition is improvised and hence the
scores associated with a composition are of limited use,
nonetheless important. The lyrics and scores, even though
not time aligned to audio recordings, are useful for computational
analysis and hence we compiled them. The primary
languages in which Carnatic music is composed are
Telugu, Tamil, Kannada, Sanskrit, and Malayalam. There
are several published compilations of lyrics and scores for
most of the currently performed compositions, such as the
ones of the three most popular composers in Carnatic music:
Tyāgarāja, Śyāmā śāstri and Muttusvāmi dīkṣitar (e.g.
[10]). However, these compilations are not machine readable
and hence not accessible for computational analysis.
There are good online open repositories for lyrics, such as
sahityam.net 11 , which is a wiki of lyrics of Carnatic compositions.
Sahityam.net is our primary source for machine
readable lyrics. It uses a uniform scheme for transliteration
to Roman script and hence has minimal ambiguity. In some
cases, it provides additional commentary, references, and
example renditions. Sahityam.net currently hosts lyrics for
about 1820 compositions of Carnatic music. Machine readable
scores are more difficult to access, with no comprehensive
machine readable score compilations available. A
set of machine readable (HTML, Word) scores compiled
by Dr. Shivkumar Kalyanaraman 12 is the main source of
scores.


The music community and music concepts related information
in the corpus form the primary source of informa-
tion for semantic analysis, and come from various sources
from the Internet. Kutcheris.com 13 is an up-to-date directory
of artist biographies, music venues, concerts and
events. The category of Carnatic music on Wikipedia 14
is a source of contextual information including music concepts.
We have added a lot of information and contributed
to Wikipedia with the help of experts. While Wikipedia
acts as an encyclopedia of music concepts providing linked
information, online music forums with discussions provide
opinions from which some of these links can be inferred.
The rasikas.org 15 Carnatic music forum is an active forum
of Carnatic music listener community with useful discussions
about Carnatic music concepts, concerts, and performances.
It is an important source of data useful for community
profiling.

-----------------------------

The main institutional reference for gathering the collection
has been the Madras Music Academy (MMA).14
This is an institution that has been hosting an annual music
festival since 1935, a festival considered the main
model for the Carnatic music tradition. The MMA has
also been hosting scholarly discussions for a long time
and these have shaped the evolution of the musical concepts
being used. The Academy has an expert committee
of veteran and contemporary artists, who formulate
and follow a procedure by which artists are selected for
the festival, criteria that has a strong influence for the
reputation of Carnatic musicians. The MMA has been
recording their concerts for a long time and its archive
is a main reference for this music tradition. However,
this archive, like most others, is not openly available online
and thus we had to rely on gathering commercial
CDs, even while following the musical criteria used by
the MMA. The main record company specialised in Carnatic
music is Charsur15. They have been publishing
high quality CDs for 15 years and thus the core of our
collection is their catalog of music concerts.
In Carnatic music, the concert, kutcheri, is a natural musical
unit that is also used as the unit for music distribution
(typically a concert is recorded in two CDs). We
have use it as the main entity in the corpus. Also in
gathering the audio collections we wanted to cover several
generations of musicians. For this, we started from
the artists that have been performing at the MMA during
the last 5 years, and then expanded the collection by including
their teachers, gurus. Table 5 shows some numbers
of the corpus. The corpus is organised in terms of
concerts, recordings (songs), artists, ragas, thalas, and
composers.
The editorial metadata has been stored and organised in
MusicBrainz.16 All the text information has been entered
in english, given that this is the language that most of the
published recordings use. We have gathered much information
about the artists in the collection, mainly from
Wikipedia and Kutcheris.com.17 We also added a lot of
information by ourselves with the help of experts.
The Carnatic tradition is very much composition based,
which is one of the main differences with Hindustani music.
Because of this, we plan to gather scores and lyrics
in order to be able to carry out some computational analysis
using them. There exists published compilations of
scores from most of the currently performed compositions,
for example the ones by the three most recognised
composers; Tyagaraja [20], Syama Sastri [21] and Dikshitar
[22]. Again the problem is to have them in a machine
readable form. Just for the lyrics there are good
online repositories, like shatyam.net.18
The Carnatic corpus has been the most worked on so far
in CompMusic, being the most complete one with respect
to the criteria identified. We are getting close to
have a corpus that can be used for many research and
application tasks. We have already done some computational
work to study the melody and rhythm of this music
by using part of this collection [18] [25] [23] [16] and we
have been able to develop a preliminary version of a web
discovery system that exploits this corpus in a practical
application [24].

\subsection{Hindustani music corpus}

\subsection{Open-access corpus}

\section{Datasets}

\subsection{Tonic dataset}

\subsection{Nyas dataset}

\subsection{Melodic similarity dataset}

\subsection{Raga recognition dataset}

\subsection{Saraga dataset}








