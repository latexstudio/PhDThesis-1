%!TEX root = ../thesis_a4.tex

\chapter{Automatic \Gls{raga} Recognition}

\section{Introduction}


As described in XXX YYY is a core musical concept used in the composition, performance, organization, and pedagogy of Indian art music (IAM). Numerous compositions in Indian folk and film music are also based on r\={a}gas~\citep{ganti2013bollywood}. Despite its significance in IAM, there exists a large volume of audio content whose r\={a}ga is incorrectly labeled or, simply, unlabeled. This is partially because the vast majority of the tools and technologies that interact with the recordings' metadata fall short of fulfilling the specific needs of the Indian music tradition~\cite{XavierSerra2011}. A computational approach to automatic r\={a}ga recognition can enable r\={a}ga-based music retrieval from large audio collections, semantically-meaningful music discovery, musicologically-informed navigation, as well as several applications around music pedagogy. 
%
%
%
%0) What is automatic raga identification
%1) Motivation and relevance of the problem
%2) References to thousands of papers that cater to this task
%3) What lacks in them
%4) How do we fill in the missing gaps
%5) What we present in this chapter
%6) What are the papers which this chapter is based on
%
%
\section{Method1: Phrase-based approach to raga recognition}

1) What is the motivation behind this method, what can be its advantages
2) What is the analogy that we have considered to devise the methodology


\section{Method2: Time delayed melodic surfaces}

1) Motivation behind this method
2) Desired qualities of this method

\section{Evaluation and Results}
\section{Comparison with state of the art}



