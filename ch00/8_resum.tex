
\chapter{Resum}

que descriu de forma automàtica el contingut de la música enregistrada és crucial per interactuar amb grans volums d'enregistraments d'àudio, i per al desenvolupament de noves eines per facilitar la pedagogia musical. La melodia és una faceta fonamental en la majoria de les tradicions de la música i, per tant, és un component indispensable en aquesta descripció. En aquesta tesi, desenvolupem mètodes computacionals per a l'anàlisi dels aspectes melòdics d'alt nivell d'actuacions musicals en la música índia de l'art (IAM), amb la qual podem descriure i connectant entre si les grans quantitats d'enregistraments d'àudio. Amb el seu marc melòdica complexa i teoria ben fonamentada, la descripció d'IAM melodia més enllà dels contorns de to ofereix un tema de recerca molt interessant i desafiant. Analitzem les melodies dins del seu context tonal, identificar patrons melòdics, comparar-los tant dins dels trossos de la música, i, finalment, caracteritzar el context melòdic específica d'IAM, el gas ra. Totes aquestes anàlisis es van realitzar utilitzant metodologies basades en dades de mida considerable corpus de música curada [TOT: futur frase perspectives].
La tesi s'inicia mitjançant la compilació i estructurar més gran fins a la data corpus de música de les dues tradicions de la IAM, Hindustani i de Carnatic música, que comprèn els enregistraments d'àudio de qualitat i les metadades associades. D'ells s'extreu el to predominant i normalitzar pel context tònica. Un element important per a descriure melodies és la identificació de les unitats temporals significatives, per la qual cosa ens proposem detectar currences ocurrència de nya s svaras de música de l'Índia, una fita que demarca els patrons melòdics musicalment més destacades.
L'ús d'aquestes característiques melòdiques, extraiem els patrons melòdics recurrents musicalment rellevants. Aquests patrons són els blocs de construcció d'estructures melòdiques, tant en la improvisació i la composició. Per tant, són fonamentals per a la descripció de les col·leccions d'àudio en IAM. Proposem un enfocament no supervisat que empra eines d'anàlisi de sèries de temps per descobrir els patrons melòdics en les col·leccions de música de mida considerable. En primer lloc, vam realitzar una anàlisi supervisat a fons de similitud melòdica, que és un component crític en el patró de descobriment. A continuació, la millora sobre el millor enfocament possible competir per explotar característiques peculiars melòdiques en IAM. Per identificar patrons significatius musicalment, explotem les relacions entre els patrons descoberts mitjançant la realització d'una anàlisi de la xarxa. Extenses proves d'escolta per músics professionals revelen que els patrons melòdics descoberts són musicalment interessant i significatiu.

Finalment, fem servir els nostres resultats per al reconeixement de gasos de l'AR en les interpretacions gravades d'IAM. Es proposen dos enfocaments nous que capturen conjuntament el tonal i els aspectes temporals de la melodia. El nostre primer enfocament utilitza patrons melòdics, les claus més importants per a la identificació ra ga pels éssers humans. Utilitzem els patrons melòdics descoberts i fem servir tècniques de modelatge tema, en el qual considerem una versió ra ga similar a una
descripció textual d'un tema. En el nostre segon enfocament, proposem el temps va retardar superfície melòdica, una característica innovadora sobre la base de coordenades de retard que captura el contorn melòdic d'un AG ra. Amb aquests plantejaments es demostra una precisió sense precedents en el reconeixement Ga ra en els conjunts de dades més grans mai utilitzats per a aquesta tasca. Encara que el nostre enfocament és guiada per les característiques de les melodies d'IAM i la tasca en qüestió, creiem que la nostra metodologia es pot estendre fàcilment a altres tradicions de la música de la melodia dominant.
En general, hem incorporat nous mètodes computacionals per a l'anàlisi de diversos aspectes melòdics d'actuacions gravades en IAM, amb el qual es descriuen i grans quantitats de Interlink d'enregistraments de música. En aquest procés hem desenvolupat diverses eines i dades que es pot utilitzar per a un nombre d'estudis computacionals amb l'IAM compilat, específicament en la caracterització de gas ra, composicions i artistes. Les tecnologies de resines ulted d'aquest treball d'investigació es parteix de diverses aplicacions desenvolupades dins el projecte CompMusic per a una millor vista general, experiència de so millorada i pedagogia en l'IAM.


\vfill
{\noindent (\emph{Translated from English by Google translator})}