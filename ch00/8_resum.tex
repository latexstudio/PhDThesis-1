
\chapter{Resum}

La descripció automàtica d’enregistraments musicals és crucial per interactuar amb grans volums de dades i per al desenvolupament de noves eines per a la pedagogia musical. La melodia és una faceta fonamental en la majoria de les tradicions musicals i, per tant, és un component indispensable per a la descripció automàtica d’enregistraments musicals. En aquesta tesi desenvolupem sistemes computacionals per analitzar aspectes melòdics d'alt nivell presents en la música clàssica de l’Índia (MCI), a partir dels quals descrivim i interconnectem grans quantitats d'enregistraments d'àudio. La descripció de melodies en la MCI, complexes i amb una base teòrica ben fonamentada, va més enllà de l’anàlisi estàndard de contorns de to (“pitch” en anglès), i, per tant, és un tema de recerca molt interessant i tot un repte. Analitzem les melodies dins del seu context tonal, identifiquem patrons melòdics, els comparem tant amb ells mateixos com amb altres enregistraments, i, finalment, caracteritzem el context melòdic específic de la música IAM: els rāgas. Tots els anàlisis s’han realitzat utilitzant metodologies basades en dades, amb un corpus musical de mida considerable.

Iniciem la tesi recopilant la col·lecció més gran de MCI obtinguda fins al moment. Aquesta col·lecció comprèn enregistraments de qualitat amb metadades de música Hindustani i Carnatic, les dues grans tradicions de la MCI. A partir d’aquí analitzem el to predominant i normalitzem la peça pel context tonal. Un element important per a descriure melodies és la identificació d’unitats temporals rellevants, per la qual cosa detectem les ocurrències de nyās svaras en la MCI, que serveixen com a marques identificadores dels patrons melòdics més destacats.

Utilitzant aquestes característiques melòdiques, extraiem els patrons melòdics recurrents més destacats. Aquests patrons són els blocs que construeixen les estructures melòdiques, tant en la improvisació i com en la composició. Per tant, són fonamentals per a la descripció de col·leccions de música MCI. Proposem partir d’un enfocament no supervisat que utilitza eines d'anàlisi basades en sèries temporals per descobrir patrons melòdics en grans col·leccions de música. En primer lloc, hem realitzat un anàlisi supervisat extensiu sobre la similitud melòdica, que és un component fonamental per al descobriment de patrons. A continuació, millorem els resultats (respecte al millor competidor segons l’estat de la qüestió) explotant les característiques peculiars dels patrons melòdics de la música MCI. Per identificar patrons musicalment rellevants, explotem les relacions entre els patrons descoberts mitjançant un anàlisi de xarxa. Extenses proves realitzades amb músics professionals revelen que els patrons melòdics descoberts són musicalment interessants i significatius.

Finalment, fem servir els nostres resultats per al reconeixement de rāgas en actuacions gravades d'IAM. Proposem dos enfocaments nous que capturen conjuntament el to i els aspectes temporals de la melodia. El primer enfoc utilitza patrons melòdics, l’aspecte més important per als éssers humans a l’hora d’identificar rāgas. Utilitzem els patrons melòdics descoberts i fem servir tècniques de modelatge de temes (“topic modeling” en anglès), on considerem que la interpretació d’un raga és similar a la descripció textual d’un tema. En el nostre segon enfocament, proposem utilitzar el “time delayed melodic surface”, una característica innovadora basada en coordenades de retard que captura l’evolució melòdica del rāga. Amb aquests enfocaments demostrem una precisió sense precedents per al reconeixement de rāgas en el conjunt de dades més gran utilitzat mai per a aquesta tasca. Encara que el nostre enfocament està basat en les característiques de les melodies MCI i la tasca en qüestió, creiem que la nostra metodologia es pot estendre fàcilment a altres tradicions de la música on la melodia és rellevant.

En general, hem incorporat nous mètodes computacionals per a l'anàlisi de diversos  aspectes melòdics per a interpretacions de MCI, a partir dels quals descrivim i inter-connectem gran quantitat d'enregistraments de música. En aquest procés hem recopilat dades i hem desenvolupat diverses eines que poden ser utilitzades per a diferents estudis computacionals per a MCI, específicament en la caracterització de rāgas, composicions i artistes. Les tecnologies resultants d'aquest treball d’investigació són part de diverses aplicacions desenvolupades dins el projecte CompMusic que pretén millorar la descripció, l’experiència auditiva, i la pedagogia de la MCI.

\vfill
{\noindent (\emph{Translated from English by Jordi Pons Puig})}