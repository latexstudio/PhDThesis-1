
\chapter{Resumen}

que describe de forma automática el contenido de la música grabada es crucial para interactuar con grandes volúmenes de grabaciones de audio, y para el desarrollo de nuevas herramientas para facilitar la pedagogía musical. La melodía es una faceta fundamental en la mayoría de las tradiciones de la música y, por lo tanto, es un componente indispensable en dicha descripción. En esta tesis, desarrollamos métodos computacionales para el análisis de los aspectos melódicos de alto nivel de actuaciones musicales en la música india del arte (IAM), con la que podemos describir y conectando entre sí las grandes cantidades de grabaciones de audio. Con su marco melódica compleja y teoría bien fundamentada, la descripción de IAM melodía más allá de los contornos de tono ofrece un tema de investigación muy interesante y desafiante. Analizamos las melodías dentro de su contexto tonal, identificar patrones melódicos, compararlos tanto dentro de los pedazos de la música, y, por último, caracterizar el contexto melódico específica de IAM, el gas ra. Todos estos análisis se realizaron utilizando metodologías basadas en datos de tamaño considerable corpus de música curada [TODO: futuro frase perspectivas].
La tesis se inicia mediante la compilación y estructurar más grande hasta la fecha corpus de música de las dos tradiciones de la IAM, Hindustani y de Carnatic música, que comprende las grabaciones de audio de calidad y los metadatos asociados. De ellos se extrae el tono predominante y normalizar por el contexto tónica. Un elemento importante para describir melodías es la identificación de las unidades temporales significativas, para lo cual nos proponemos detectar currences ocurrencia de nya s svaras de música de la India, un hito que demarca los patrones melódicos musicalmente más destacadas.
El uso de estas características melódicas, extraemos los patrones melódicos recurrentes musicalmente relevantes. Estos patrones son los bloques de construcción de estructuras melódicas, tanto en la improvisación y la composición. Por lo tanto, son fundamentales para la descripción de las colecciones de audio en IAM. Proponemos un enfoque no supervisado que emplea herramientas de análisis de series de tiempo para descubrir los patrones melódicos en las colecciones de música de tamaño considerable. En primer lugar, realizamos un análisis supervisado a fondo de similitud melódica, que es un componente crítico en el patrón de descubrimiento. A continuación, la mejora sobre el mejor enfoque posible competir por explotar características peculiares melódicas en IAM. Para identificar patrones significativos musicalmente, explotamos las relaciones entre los patrones descubiertos mediante la realización de un análisis de la red. Extensas pruebas de escucha por músicos profesionales revelan que los patrones melódicos descubiertos son musicalmente interesante y significativo.

Por último, utilizamos nuestros resultados para el reconocimiento de gases de la AR en las interpretaciones grabadas de IAM. Se proponen dos enfoques novedosos que capturan conjuntamente el tonal y los aspectos temporales de la melodía. Nuestro primer enfoque utiliza patrones melódicos, las claves más importantes para la identificación ra gA por los seres humanos. Utilizamos los patrones melódicos descubiertos y empleamos técnicas de modelado tema, en el que consideramos una versión ra gA similar a una
descripción textual de un tema. En nuestro segundo enfoque, proponemos el tiempo retrasó superficie melódica, una característica novedosa sobre la base de coordenadas de retardo que captura el contorno melódico de un AG ra. Con estos planteamientos se demuestra una precisión sin precedentes en el reconocimiento gA ra en los conjuntos de datos más grandes jamás utilizados para esta tarea. Aunque nuestro enfoque es guiada por las características de las melodías de IAM y la tarea en cuestión, creemos que nuestra metodología se puede extender fácilmente a otras tradiciones de la música de la melodía dominante.
En general, hemos incorporado nuevos métodos computacionales para el análisis de varios aspectos melódicos de actuaciones grabadas en IAM, con el que se describen y grandes cantidades de Interlink de grabaciones de música. En este proceso hemos desarrollado varias herramientas y datos que se puede utilizar para un número de estudios computacionales con el IAM compilado, específica- mente en la caracterización de gas ra, composiciones y artistas. Las tecnologías de resinas ulted de este trabajo de investigación se parte de varias aplicaciones desarrolladas dentro del proyecto CompMusic para una mejor descripción, experiencia de sonido mejorada y pedagogía en el IAM.

\vfill
{\small \noindent (\emph{Translated from English by Google translator})}