% The list of symbols is a chapter, but without a number
\chapter*{List of Symbols} 
\addcontentsline{toc}{chapter}{List of Symbols}
The following is a list of different symbols used in the dissertation along with a short description of each symbol. 

% Define symbol definitions
\newcommand\listSymbol[3]{\protected\gdef#1{#2}#2 & #3 \tabularnewline \addlinespace[2pt]} 

% Defined and listed
\newcommand\nolistSymbol[3]{\protected\gdef#1{#2}} % Defined, but not listed

\begin{longtable}{P{15mm}>{\raggedright}p{85mm}}
	\toprule
	\textbf{Symbol} & \textbf{Description} \tabularnewline \midrule
	\endhead % all the lines above this will be repeated on every page
	\listSymbol{\pitchHz}{\ensuremath{p}}{Predominant pitch}
	\listSymbol{\sRate}{\ensuremath{S}}{Sampling rate of the pitch sequence (Hz)}
	\listSymbol{\mNorm}{\ensuremath{N}}{Normalization type used in the melody representation}	
	\listSymbol{\uTScaling}{\ensuremath{U}}{Uniform time-scaling factor}	
	\listSymbol{\distPatt}{\ensuremath{D}}{Distance measure for computing melodic similarity}
	\listSymbol{\pVal}{\ensuremath{\rho}}{p-value in statistical hypothesis testing}	
	\listSymbol{\nSvara}{\ensuremath{\mathcal{N}}}{Ny\={a}s svara segment}											%This is N in figures
	\listSymbol{\freqSvara}{\ensuremath{\mathcal{s}}}{Svara frequency (Cents)}										%this is S in figures
	
	
	
\end{longtable}	