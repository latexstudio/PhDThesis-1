Technologies that can `make sense' of digital information such as digital music is now a days a necessity in modern world. Such technologies can be used to develop novel ways to interact with massively growing music content and to build innovative tools for music education and music creation. In most musics of the world melody is a fundamental dimension of music and its computational analysis is a core XXX for XXX. While there are several transversal aspects of melody across music traditions, there are significantly varied culture specific aspects that pose unique challenges to existing MIR approaches and call for a novel methodologies to address them. Focusing on such specificities can expand the existing knowledge and make MIR technologies more complete and versatile. 

In this thesis we aim to develop computational approaches for automatic description and analysis of melodies in \gls{iam}. Our data-driven approaches are based on signal processing, time-series analysis and machine learning concepts. We start with identifying relevant research problems that address the challenges, and benefit from the opportunities offered by the complex melodic structures in \gls{iam}. To ensure real-world scalability of our approaches, building representative data corpora and datasets has been one of the focus throughout the thesis. Recurring pattens are fundamental units in melodies of IAM, and therefore, they become the central XXX utilized for analyzing melodies in this study. Different approaches addressing relevant pattern processing tasks such as melodic similarity, patten discovery and pattern characterization are proposed. Our quantitative and qualitative evaluations show promising results. The music recordings, performances, pedagogy in IAM is structured around the concept of raga. We address the fundamental task of automatic raga recognition and propose two approaches. Our first approach utilize discovered melodic patterns for recognition ragas, wherein we borrow the concepts from topic modeling. Our second approach is based on a novel melodic representation that captures both tonal and temporal characteristics of melody. With our extensive evaluations and comparisons with the existing methods we show that we advance the state of the art in raga recognition. Data pre-processing? We presenting a number of applications built on top of our approaches that demonstrate the usefulness and relevance of the output of our approaches. 



%
%%%%%%%%%%%%%%Applications of the technologies we work on: Organization, Navigation, Recommendation, Discovery, Creation, Education, Evaluation, Enhancements (listening, performance), Musicology, Other studies
%
%%%%%% What actions can be done automatically to give rise to applications above
% ()Keywords) automatic description, indexing, search, retrieval, interaction with music content 
%
%%%%%%%%%%%%%Applications/motivations of our specific tasks
%
%Tonic Identification: understanding musical concepts (drone + raga-tonic dualities etc), automatic music description, input to higher level analysis
%
%Nyas segmentation: Understanding musical concept, automatic description, input to other analyses, enhanced listening, education tool
%
%Similarity: understanding musical concept, automatic description, input to other analyses, enhanced listening, music education tools, establishing similarities and influences between artists, school of music, ragas, and recordings.... input to higher level analysis, navigation and discovery, musicological analysis, similarity based retrieval
%
%discovery: understanding musical concepts, understanding improvisation in IAM, understanding creative aspects of IAM, music generation, enhanced listening, music education tools, establishing similarities and influences between artists, school of music, ragas and recordings...input to higher level analysis, navigation and discovery, musicologial analysis, indexing, search and retrieval, similarity based retrieval
%
%Characterization: al that discovery does + understanding function roles of differnt type of patterns, 
%
%Raga recognition: understaing music, automatic description, education tool, establishing sim and inf like said above, raga based music retrieval, organization, navigation and discovery, indexing, searhc and retrieval 

%%%%%%%%%%%%%%% Different scientific areas that we use the concepts from
% Signal processing, time-series analysis, machine learning, musicology

%%%%%%%%%%%%%% Our approaches/work (adjectives) 
% Data-driven, involve both top-down and bottom-up approach, Culturally aware, human interpretable stages and results and learnings, applied research, focus on understanding than numbers, quantitative and qualitative evaluations, Knowledge driven??, domain-specific, content-based, mostly using audio signal information, 

%%%% Main goals
% Select relevant (core) problems, gather representative data, understand choices of parameters and processing steps since characteristics are so diff, understanding of challenges and opportunities in diff tasks, influence of characteistics on choices of procedure and params, influence of params and procedures on tasks, comparative eval whereever possible, compile literature, Evaluate and verify hypothesis and approaches, exploiting cultural specificities to advance!!! identify key melodic unit to describe melodies..exploiting specificities of musical culture..melodic pattern similarity, discovery and characterization, unsupervised analysis, bringing novel insights, large-scale study,  raga-recognition, quantitative evaluation of musical relevance of discovered patterns, novel melodic representation, pushing state of the art in raga recognition..taking analogies with topic modeling techniques, connection between describing a topic and rendering a raga. open data, open code, reproducibility, applications, demonstrating usefulness and potential of such technologies, 


