\chapter*{Acknowledgements}

When I embarked on this journey of pursuing a PhD in music technology, I thought of it as solely an intellectual pursuit, but very soon I realized that it is a way of life. A life that gets enriched by the people we engage with, and I am extremely grateful to many people for making this experience a memorable one. My heartfelt thanks to my thesis supervisor Prof Xavier Serra, whose vision of a world that nurtures different cultures led to the genesis of the CompMusic project. I thank him for giving me an opportunity to work with immense creative freedom, as well as his supervision and support throughout the dissertation. His ability to never lose sight of the big picture combined with his eye for detail, his leadership-style and magnanimity are life-lessons that shall stay with me in all my future pursuits.

I thank Joan Serrà for his constant guidance and his tireless strive for perfection, some of which, I hope, I imbibed in the process. His ideas and feedback have been instrumental in shaping my work at every step, as well as my outlook toward research work in general. His friendship has encouraged and inspired me in the good and the not-so-good times. 

I would like to thank Prof Preeti Rao for her guidance during my initial years in the field, and Prof Hema Murthy for her collaboration and support. I would also like to thank Emilia Gómez and Perfecto Herrera for their valuable inputs at crucial junctures.  

I thank Cristina Garrido, Sonia Espí, Alba B Rosado, Vanessa Jimenez, Lydia García, Marcel Xandrí and Jana Safrankova for always being ready to support me with a smile and helping me untangle the infinite web that bureaucracy could be. 

I gratefully acknowledge Joe Cheri Ross, Vinuta Prasad, Shrey Dutta, Ashwin Bellur, and Ranjani HG for their collaboration and valuable inputs. I would also like to thank Vignesh Ishwar and Kaustuv K. Ganguli, my friends and gurus in Carnatic and Hindustani music, respectively, for not only collaborating with me in my work given their expertise in music but also for bringing the magic of Carnatic and Hindustani music to my life here in Barcelona through their wonderful performances both on and off stage.

I am extremely grateful to the friendships that were forged during this time in the MTG. I thank Ajay Srinivasamurthy for being an epitome of generosity, Marius Miron, for being such an amazing person and all his help and support, Sertan \c{S}ent{\"u}rk and Gopala K. Koduri for being awesome conversationalists and co-workers, Rong Gong whose indefatigable spirit is a thing to emulate, Rafael Caro Repetto whose patience and calmness could make Yogis introspect, Georgi Dzhambazov (still can't pronounce his last name) for his insatiable curiosity, Swapnil Gupta for being my music partner and stepping in to be my room-mate and helping me out, Alastair Porter for knowing everything about everything in technology and Andres Ferraro for helping me out and always being there for any technical support. This journey has been made further memorable by my colleagues Justin Salamon, Mohamed Sordo, and Frederic Font whose comments and suggestions have helped me look at my work from new perspectives. 

I thank Jose Zapata, Pauli Be, Marius Miron and Ajay Srinivasamurthy, my wonderful flat-mates, who gave me a home away from home. My friends, Sergio Giraldo, Varun Jewalikar, Hector Parra , Dara Dabiri, Aluizio Oliveira, Nadine Kroher, Juanjo Bosch and Sergio Oramas for all the fun-times and memories that we made during the last four years and for the sheer brilliance of their beings. My special thanks to Eva Arrizabalaga, whose kindness makes this world a little better everyday.

Last but not the least, I would like to thank my parents for always believing in me and being a constant source of support and strength. My siblings, Vikram and Ruchi for the joy they bring to my life, and my life-partner Shefali, for always being there for me...


%
%\vspace*{\fill}
%
%\line(1,0){372}\\
%\footnotesize
%This thesis has been carried out at the Music Technology Group of Universitat Pompeu Fabra (UPF) in Barcelona, Spain, from Oct.~2012 to Sep.~2016. This work has been supported by the Dept. of Information and Communication Technologies (DTIC) PhD fellowship (2012-16), Universitat Pompeu Fabra and the European Research Council under the European Union’s Seventh Framework Program, as part of the CompMusic project (ERC grant agreement 267583).
\normalsize