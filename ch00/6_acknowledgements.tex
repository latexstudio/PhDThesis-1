\chapter*{Acknowledgements}

\TODO{Use full names!!}
When I embarked on this journey of pursuing a PhD in Music Technology, I looked at it as solely an intellectual pursuit but very soon in I realised it is a way of life. A life that gets enriched by the people we engage with and I am extremely grateful to many people for making this experience a memorable one. My heartful thanks to my thesis supervisor Prof Xavier Serra, whose vision of a world that nurtures different cultures led to the genesis of the CompMusic project. I thank him for giving me an opportunity to work with immense creative freedom as well as his supervision and support throughout the dissertation. His ability to never lose sight of the big picture combined with his eye for detail, his leadership-style and magnanimity are life-lessons that shall stay with me in all my future pursuits.

I thank Joan Serra for his constant guidance and his tireless strive for perfection, some of which, I hope, I imbibed in the process. His ideas and feedback have been instrumental in shaping my work at every step as well as my outlook toward research work in general. His friendship has encouraged and inspired me in the good and the not-so-good times. I would like to thank Prof Preeti Rao for all her guidance during my initial years in the field and Prof Hema-- for her collaboration and support. I would also like to thank Emilia Gomez, Perfecto Herrera for their  valuable inputs at crucial junctures.

I thank Cristina, Sonia, Alba, Vanessa, Lydia, Judith and Jana for always being ready to support me with a smile and helping me untangle the infinite web that bureaucracy could be. This journey has been made further memorable by my colleagues Justin, Moha, and Frederic whose comments and suggestions have helped me look at my work from new perspectives. 

I gratefully acknowledge Ross, Vinuta, Shrey, Ashwin and Ranjani for their collaboration and valuable inputs . I would also like to thank Vignesh and Kaustuv, my friends and gurus in Carnatic and Hindustani music respectively, for not only collaborating with me in my work given their expertise in music but also for bringing the magic of Carnatic and Hindustani music to my life here in Barcelona through their wonderful performances both on and off stage.

I am extremely grateful to the friendships that were forged during this time in the MTG lab. I thank Ajay for being an epitome of generosity, Sertan and Gopal for being awesome conversationalists and co-workers, Rong whose indefatigable spirit is a thing to emulate, Rafael whose patience and calmness could make Yogis introspect, Georgi for his insatiable curiosity, Swapnil for being my music partner and stepping in to be my room-mate, Alastair for knowing everything about everything in technology and Andres for always being available for any technical support.

I thank Jose, Paula, Marius and Ajay, my wonderful flatmates, who gave me a home away from home. My friends, Sergio, Varun, Hector, Dara, Aluizio, Nadine, Juanjo, Sergio for all the fun-times and memories that we made during the last four years and for the sheer brilliance of their beings. My special thanks to Eva, whose kindness makes this world a little better everyday.

Last but not the least, I would like to thank my parents for always believing in me and being a constant source of support and strength. My siblings, Vikram and Ruchi for the joy they bring to my life, and my partner Shefali, for always being there for me through all the good and \st{the bad} not so good times...



\vspace*{\fill}

\line(1,0){372}\\
\footnotesize
This thesis has been carried out at the Music Technology Group of Universitat Pompeu Fabra (UPF) in Barcelona, Spain, from Oct.~2012 to Sep.~2016. This work has been supported by the Dept. of Information and Communication Technologies (DTIC) PhD fellowship (2012-16), Universitat Pompeu Fabra and the European Research Council under the European Union’s Seventh Framework Program, as part of the CompMusic project (ERC grant agreement 267583).
\normalsize