%!TEX root = ../thesis_a4.tex

\chapter{Summary and Future Directions}
\label{sec:summary_future_work}

\section{Summary of the Thesis}
\section{Summary of Contributions}
\section{Future Directions}



I am writing future work and possible conclusions in random order as they come 

We wont go very far with an unsupervised approach without the mechanism of evaluating it. But if the annotation process was not tricky we would be using supervised. So a balanced approach would be to use discovery methods to generate evaluatino set and bootstrap the process. 

Phrase boundary extension collectively within a community of phrases...to study a common defining length

Exploiting melodic context in discovering and detecting phrases

Exploiting if melodic characteristic of he phrase can tell us something about the phrase category

Incorpotating segmentation models or using methods that does not require segmentation.

In the future we plan to include other aspects of melody such as loudness and timbre in the similarity computation and use a bigger dataset consisting of many more melodic patterns for evaluation.
.

In the future, we plan to investigate if both these methodologies can be successfully combined to improve r\={a}ga recognition.

Hierarchical raga recognition model

Temporally minimum duration needed

INdexing techniques.