%!TEX root = ../thesis_a4.tex

\chapter{Summary and Future Perspectives}
\label{chap:summary_future_work}

\section{Introduction}
\label{sec:summary_thesis}

 In this thesis we have presented computational approaches for analyzing a number of melodic elements at different levels of melodic organization in \gls{iam}, which in tandem generate a higher-level melodic description of its audio collections. These approaches build upon each other to finally allow us to achieve the goals that we set at the beginning of this thesis, which are: 
 
 \begin{itemize}
 	\item To curate and structure a representative music corpora of \gls{iam} that comprise audio recordings and associated metadata, and use that to compile sizable and well annotated tests datasets for melodic analyses.
 	\item To develop data-driven computational approaches for discovery and characterization of musically relevant melodic patterns in sizable audio collections of \gls{iam}
 	\item To devise computational approaches for automatically recognizing \glspl{raga} in recorded performances of \gls{iam}.
 \end{itemize}
 
 Based on the results presented in this thesis we can now say that our goals are successfully met. In~\chapref{chap:corpus_music_corpora_and_datasets} we provided a comprehensive overview of the music corpora and datasets that were compiled and structured in our work as a part of the CompMusic project. To the best of our knowledge these corpora comprise the largest audio collections of \gls{iam} along with curated metadata and automatically extracted music descriptors that is available for research. Furthermore, the datasets that we built from these corpora allowed us to successfully evaluate our computational approaches, with some of them being the largest datasets ever used for such evaluations. 
 
 In~\chapref{chap:melodic_pattern_processing} we argued that the potential of a pattern-based melodic analysis in \gls{iam} can be exploited using an unsupervised approach to extract melodic patterns from audio recordings. We demonstrated that using our novel approach we can successfully discover musically significant melodic patterns in hundreds of hours of audio collections of \gls{iam} without requiring any exemplars of melodic patterns from experts as input.
 
 In~\chapref{chap:raga_recognition} we described two novel approaches for \gls{raga} recognition that jointly utilize the tonal and the temporal characteristics of melodies in \gls{iam} without discretizing the melody representation. We demonstrated that our approach can effectively utilize the discovered melodic patterns for \gls{raga} recognition. We also presented our approach that abstracts continuous melody representation to capture the melodic outline relevant for characterizing \glspl{raga}. Using this approach we demonstrated unprecedented accuracies in the task of \gls{raga} recognition using largest datasets ever used for this task. 
 
 We note that the approaches we propose to perform these tasks are not 100\% accurate, and there exists a large scope for improvement. However, a significant part of the errors made by our approaches can be explained from a musicological and perceptual perspective. The cases where the system fails are often the ones which are challenging even for a human listener. 
 
 We started this thesis by presenting our motivation behind the analysis and description of melodies in \gls{iam}, highlighting the opportunities and challenges that this music tradition offers within the context of music information research and the compmusic project (\chapref{chap:intro}). We provided a brief introduction to \gls{iam} and its melodic organization,  and reviewed the existing literature on related topics within the context of \gls{mir} and \gls{iam} (\chapref{chap:background}). We described the music corpora of \gls{iam} and different test datasets, emphasizing the design criterion taken in the Compmusic project to compile and curate the corpora (\chapref{chap:corpus_music_corpora_and_datasets}).  We described and evaluated the approaches we follow to obtain melodic descriptors and melody representations with which we perform melodic analysis in this thesis (\chapref{chap:data_preprocessing}). We then presented and evaluated our approaches for computing melodic similarity, discovering melodic patterns and characterizing melodic patterns (\chapref{chap:melodic_pattern_processing}). Finally, we presented and evaluated our novel approaches for \gls{raga} recognition that utilize discovered melodic patterns and abstracted melody representation to outperform the state-of-the-art (\chapref{chap:raga_recognition}).
 
 At the end of each chapter in this thesis we included a section that summarized the key results and conclusions of the work reported in that chapter. We here provide an overall summary of the thesis, highlight our main contributions (\secref{sec:summary_contributions}), and finally, end the thesis with some discussions about the future perspectives (\secref{sec:future_directions}). 
 
 
\section{Summary of Contributions}
\label{sec:summary_contributions}
We now present a summary of the main contributions of this thesis.

\subsubsection*{Contributions to Creating Music Corpora and Datasets}

One of the objectives of the CompMusic project was to build a quality research corpora of \gls{iam}, with which to study different computational tasks in the context of \gls{mir}. The task of compiling and curating the research corpora and different test datasets has mainly been a team effort. We describe below some specific contributions from the author. 

\begin{itemize}
	
	\item The contributions to compiling corpora are mainly in Hindustani music corpus (\secref{sec:corpus_hindustani_music_corpus}), starting from the procurement of music CDs, ripping and structuring the audio collection and manually adding all the editorial metadata to MusicBrainz. 
	
	\item Compiling CompMusic Tonic Datasets (\acrshort{tds_cm1}, \acrshort{tds_cm2} and \acrshort{tds_cm3}), which collectively include tonic annotations for 716 audio recordings spanning. \TODO{Double check this number}
	
	\item Compiling \Gls{nyas} dataset (\acrshort{nds_cm}), which includes annotations of \gls{nyas} segments for 20 audio recordings spanning 1.5\,hours of audio, done in collaboration with Kaustuv Kanti Ganguli. 
	\item Improving Melodic Similarity Dataset (\acrshort{msds}), which originally included 497 annotated instances of 10 different melodic patterns in 33\,audio recordings done by Kaustuv Kanti Ganguli and Vignesh Ishwar. In the revised version of the dataset that came after a detailed verification, 127\,instances of melodic patterns were added for the same audio recordings and pattern categories. 
	\item Compiling \Gls{raga} recognition datasets (\acrshort{rrds_cmd_big} and \acrshort{rrds_hmd_big}), which contain \gls{raga} labels and associated editorial metadata for 780\, audio recordings, done in collaboration with Vignesh Ishwar and Kaustuv Kanti Ganguli. \acrshort{rrds_cmd_big} comprise 480\,full length recorded performances in 40\,\glspl{raga} with 12 recordings per \gls{raga} spanning a total of 124\,hours of audio. \acrshort{rrds_hmd_big} comprise 300\,full length recorded performances in 30\,\glspl{raga} with 10 recordings per raga spanning a total of 130\,hours of audio.
\end{itemize}

\subsection*{Scientific and Technical Contributions}

\begin{itemize}
	\item Review of current approaches for tonic identification, melodic pattern processing and \gls{raga} recognition in the context of \gls{mir} in \gls{iam}, emphasizing their limitations and identifying avenues for scientific contributions.
	\item Comprehensive assessment of different tonic identification approaches on a number of sizable datasets,  along with a detailed error analysis for different types of music material.
	\item Development of a novel \gls{nyas} landmark-based approach for the segmentation of melodies in Hindustani music. \Gls{nyas} detection is addressed for the first time from a computational perspective.
	\item In depth evaluation of different procedures and parameter settings for computing melodic similarity in the context of short-duration \gls{raga} motifs in \gls{iam}. 
	\item A partial transcription and complexity weighting-based approach for improving melodic similarity is devised that exploits peculiar characteristics of melodies in \gls{iam} .
	\item Demonstration of the utility of an unsupervised approach for discovering repeated melodic patterns in sizable audio collections of \gls{iam}.
	\item Characterization of the discovered melodic patterns by employing network analysis tools to identify musically significant patterns, the \gls{raga} motifs. 
	\item Demonstration of the utility of a novel pattern-based approach for \gls{raga} recognition, which employs vector space modeling concepts to exploit discovered melodic patterns for this task. 
	\item Development of  a novel feature, the \acrfull{tdms}, which captures both the tonal and the short-time temporal characteristics of a melody. This feature together with a simple \gls{1nn} classification strategy is shown to outperform state of the art in \gls{raga} recognition using the largest datasets ever used for this task. 
\end{itemize}

Most of the outcomes of the work presented in this document have been published in the form of papers in international conferences, and journals. The full list of the author’s publications is provided in~\appref{app:publications}. The code and tools developed during our work are made publicly available to facilitate reproducible research and comparative studies (). The set of tools and output of our approaches are also integrated in Dunya, a system that consolidates the corpora and computational tools for XXX. There are a number of online Web demos built to demonstrate the usefulness of the outcome of our approaches, which are discussed in~\secref{}. Furthermore, we have also exploited a number of the outcomes of our work for developing mobile applications that provide enhanced listening experience, and novel pedagogical tools in the context of \gls{iam} \secref{}.

\TODO{Any overall big sentence, that we have opened out research area? Good datasets and spectrum of problems.}

\section{Future Perspectives}
\label{sec:future_directions}

There are several directions for future work in both improving the methodologies described in this thesis, and addressing unexplored research problems that can utilize our results and resources. The music corpora compiled and curated in the CompMusic project provides a strong base to address a variety of such research problems in the future.

We start with enumerating different avenues for improvement in the patten processing methodologies discussed in our work. In~\chapref{chap:melodic_pattern_processing} we argued that the potential of pattern-based analysis and description of melodic aspects in \gls{iam} can be exploited by going beyond supervised methodologies for pattern processing. We successfully demonstrated the effectiveness and utility of an unsupervised approach for discovering melodic patterns. However, lack of a quantitative assessment of such an approach hampers its improvement. We believe that a balanced combination of both supervised and unsupervised methodologies would take pattern processing in \gls{iam} to the next level. One of the ways in which both these approaches can be combined is by using the output of the unsupervised approach for facilitating annotations of large amounts of melodic patterns. The biggest limitations of the supervised approaches as mentioned in~\secref{sec:patterns_introduction} are the difficulties in annotating large datasets and the presence of human bias in the annotations, resulting in only certain type of melodic patterns. Both these issues can be addressed to a large extent by utilizing the outcome of pattern discovery approach for annotations.

With sizable annotated datasets of melodic patterns several valuable insights into perception of melodic similarity can be obtained. By sizable dataset we mean thousands of melodic patterns across different artists, compositions, forms, and \glspl{raga}. We here provide some concrete examples of analyses that can be done with such datasets. One of our learnings while working on the melodic similarity is that not every sample in the string representation of a melodic pattern contributes equally in establishing similarity. There are specific regions and characteristic pitch movements in the melodic patterns that are more important and often become the anchor points in determining similarity. Note that this observation is based on our informal discussions with musicians and this phenomenon is yet to be scientifically studied. A sizable corpora of melodic patterns can facilitate such investigations, which in turn would improve models for computing melodic similarity. Another interesting and important aspect to explore in pattern discovery is to take into account the local melodic context of a pattern. Since melodies in \gls{iam} are constructed in accordance with the \gls{raga} grammar, \gls{raga} motifs might bear a strong correlation with their melodic context, which can be exploited it to improve pattern discovery and and reduce its computational complexity. In addition, determining possible relation between the \gls{sama} locations (downbeat in rhythm cycle) and the locations of \gls{raga} motif is also an interesting subject of future investigations. All these analyses can benefit immensely from a well annotated sizable dataset of melodic patterns. 

In our work we considered the predominant pitch in the audio recording as the low-level melody representation. However as illustrated by an example in  XXX timbral and loudness characteristics of melody are also important in establishing similarity. Incorporating those dimensions of melody into similarity computation is also something to be explored in the future.

One of the negatives about our methodology is the ccompuational complexity. Since it was the first time when melodic patterns were mined at sucha large scale we did not want to impose top down knowledge to not loose any possible interesting patterns. However there can be several strategies applied to make it efficient. We can avoid sliding window awy, only consider those landmarks whih are possible stating points of patterns. It depends on the accuracy of segmentation method, but some obvkous things like not starting a pattern from the middle of a note is something we can try out. Another thing is to explore locality sensitive hashing techniques to make it faster. One kind of top down knowledge cna be incorporated in the form of a meldy transcription stage. By a reliable rich transcriptino of melodic nuances, if they can be converted to meaningful symbols it can improve melodic similarity manifold. The challenge here is to reliable detect and encode melodic ornaments and other melodic moements. Such an approach will not only make thecoputation faster but it will also make the similarity more meaningful. Also we can parametrically control the contribution of different types of symbols in the similarity computation. 

As we saw from our results in XXX, a meaningful segmentaion of melodic pattens affects tremendously the accuracy of melodic similarty. Thus a reliable phrase segmentation can improve the system greatly :). We explored nyas based system, however we find that it is mainly applicable in slow tempo Hindustani melody. a more generic and transversal model is needed. This aspect is studied a lot for symbolic music case and is completely unexplored in the case of recorded performances. It is a promising direction of future work. Segmentation can benefir fro out work. There are examples of sementatino using parallelism. Since we can't formulate anything otherwise this can help us understand segmentatino issues. 

Another limitation of our current approach is that we work with fix duration patterns. This was mainly due to the contraints posed by lower bounding methods and also without a segmentation model detecting such boundaries is not possible?. However, once the patterns are discovered there can be some work done to improve the boundaries of these patterns. For eample considering patten pairs and extending their boundaries till they are similar in melodies. Eventually utilizing multiple patterns of the same category (in a category) for this task can help ideantify meaningful segmentatino boundaries of the patterns. This step will also make ourpattern variable duration motifs. say that formally studying boundaries is a complex task. In fact segmentation can be studied using this method!!! yes

In our charaterization step we discover that several patterns are gaaka typee paters. While at several step we followed a philosophy of removing trivial pattens at the pre-processing step. We removed tani patterns and flat regions in melodies. A similar thing can be done with gamaka type patterns since they follow an explicit model/shape it shouldbe easy to detect them beforehand. This would make our process more efficient and more probability of obtaining meaninful ptterns, In characteriation Specific improvements mentioned in the text, like considering compositions instead of recordings for improving goodness measure. Differnt community detection methods can be tried out while network anlaysis. We did not explore much. 


Now going to raga recognition approaches. 

5) Raga recognition

Ragawise a good model

Study minimum duration needed to identify. Crucial in real-time applications. Due to linearity in exploration of svaras PCD will fail miserably

Heirarchical model





New analyses
0) Characterization based on melodic features
1) music similarity across recordings
2) Corpora level characterization 
3) Evolution of a music piece, discovery of melodic structures!!
4) All these melodic elements combined in music ontology and knowledge databases


%4) Pattern discovery - improving quality of the output patterns
%
%Boundaries of the pattern, extend ---however defining it formally is a complex task, since its always sung in a particular context, how to differentiate the context from the patter?
%Boundaries will improve similarity, and improve the quality
%Improving similarity -> improve this...
%trying other community detection methods
