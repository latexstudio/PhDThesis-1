%!TEX root = ../thesis_a4.tex

\chapter{Summary and Future Perspectives}
\label{sec:summary_future_work}

\section{Introduction}
\label{sec:summary_thesis}

In this thesis we have presented computational approaches for analyzing a number of elements/concepts/components at different levels of melodic organization in \gls{iam}, which in tandem generate a higher-level melodic description of its audio collections. These approaches build upon each other to finally achieve the goals that we set at the onset of this thesis, which are: 

\begin{itemize}
	\item To build a representative music corpora of \gls{iam} that comprises audio recordings and the associated metadata, and use that to compile sizable and well annotated tests datasets for melodic analyses.
	\item To devise a data-driven computational approach to discover musically relevant melodic patterns in sizable audio collections of \gls{iam}.
	\item To devise culture-aware approaches for automatic \gls{raga} recognition.
\end{itemize}

finally address the broad level research questions that guided our work, which are; Can a machine discover and characterize melodic pattens in sizable audio collections of \gls{iam} without any exemplar patterns?, Can we use these patterns to recognize \glspl{raga} in recorded performances of \gls{iam}?  Can we effectively combine tonal and temporal aspects of melodies to improve state-of-the-art in \gls{raga} recognition? Based on the results presented in this thesis we can now say, yes. Of course the accuracies we obtain in these tasks are yet not 100\% and depend on the complexity of the task. Moreover, tasks such as melodic pattern discovery and raga recognition are not trivial for a human listener. These tasks are performed reliably by trained listeners (avid listeners or musicians). One of the goals of this thesis was to open up the possibilities of different computational tasks in INdian art music. With the promising results we obtain and the potential areas of further improvement we identify we believe that our work will be helpful towards advancing mir for IAM. 


We start with providing an introduction and context to the work presented in this thesis (). We highlight the opportunities and challenges that melodic description of \gls{iam} offers. In chapter 2 we present a critical revire of the existing literature on relevant topics for IAM. We also review related tasks done for other music traditions. IN addition we provide musical backgrund to appreciate the musical context of the work. In chapter 3 we sujmmarized the music corpous that we compile and curate for working in this thesis. We summarized our main criterions and described specific test datasets used for different tasks in this thesis. In chapter 4 we describe the approaches we follow for extraction of tonal context, low-level predominant pitch contours, nyas segments in melodies of HIndustani music and tani sgments in carnatic music. With these descriptors and representation and  information about the different melodic aspects we proceed to discover musically significant melodic patterns in audio colletions of IAM. We study melodic similarity, improve melodic similarity, describe our approach for discovery, and finally characterize the discovered melodic oattersn. Using these melodic patterns we build raga recognition method. We also propose altermnate method that abstract tonal temporal and surpases sota. We demo applications of our work in different contexts. And finally summarize the thesis. 


We started with an intro 
We did critical review 
We described our corpora, building strategy and datasets
We described our approaches for extracting melodic descriptors and representations that enable higher level melodic analyses
We described approach for pattern discovery
We described our approach for raga recognition.
We provided applications that can be built with this resaerch
We conclude this thesis with summary, contributions and some future perspectives.




\section{Summary of Contributions}
\label{sec:summary_contributions}

Music Corpora and Datasets

Corpora compiling and curating.

Test Dataset compiling and curating.

Scientific and Technical Contributions

1) Literature review
2) TOnic id eval
3) Nyas id novel approach
4) Similarity comparative eval
5) Approach to improve similarity
6) Discovery and characterization of pattern
7) Novel approaches for raga recognitino Improving state of the art in raga rec





Share datasets, code etc

\section{Future Directions}
\label{sec:future_directions}

Lots of them!

Overall broad level approach to melodic analysis.
Particular improvements in specific approaches..




%
%I am writing future work and possible conclusions in random order as they come 
%
%We wont go very far with an unsupervised approach without the mechanism of evaluating it. But if the annotation process was not tricky we would be using supervised. So a balanced approach would be to use discovery methods to generate evaluatino set and bootstrap the process. 
%
%Phrase boundary extension collectively within a community of phrases...to study a common defining length
%
%Exploiting melodic context in discovering and detecting phrases
%
%Exploiting if melodic characteristic of he phrase can tell us something about the phrase category
%
%Incorpotating segmentation models or using methods that does not require segmentation.
%
%In the future we plan to include other aspects of melody such as loudness and timbre in the similarity computation and use a bigger dataset consisting of many more melodic patterns for evaluation.
%.
%
%In the future, we plan to investigate if both these methodologies can be successfully combined to improve r\={a}ga recognition.
%
%Hierarchical raga recognition model
%
%Temporally minimum duration needed
%
%INdexing techniques.