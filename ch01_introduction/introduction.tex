%!TEX root = ../thesis.tex

\chapter{Introduction}
\label{chap:intro}

\section{Motivation}
\label{sec:intro_motivation}

Repeating structures are important information units in data such as text, DNA sequences, images, videos, speech and music~\citep{Buhler2002b,Herley2006}. Patterns are exploited in a variety of ways, ranging from signal level tasks such as data-compression~\citep{Atallah1999} to more cognitively complex tasks such as analyzing an art work~\citep{van2010texton}. In music domain, identification of repeating structures in a musical piece is fundamental to its analysis, understanding and interpretation~\citep{Cook1987,Lerdahl1983}. 



\subsection{Music Information Retrieval}
\label{sec:intro_motivation_mir}

\begin{itemize}
	\item What is music information retrieval
	\item What is the context in which music information retrieval is growing
	\item Why is MIR getting more important and finding many applications
	\item Successful examples of MIR
	\item What are the different contexts related with music that MIR can help in
	\item what more?
\end{itemize}

\subsection{CompMusic Project}
\label{sec:intro_motivation_compmusic}

\begin{itemize}
	\item MIR is important and things are going fine, what's the issue then? Is there any bias in the kind of music tradition being analyzed in MIR right now? whats the reason for this bias
	\item What kind of music traditions are kind of ignored for anlaysis now?
	\item What's the reason why no one wants to consider them?
	\item Is it even relevant to consider them? Which ones amongst them are the easiest ones or the most relevant ones to consider
	\item Does everyone get benefitted from this kind of work? how things improve overall by doing this?
	\item What is the objective of COmpMusic project, which music traditions it focuses on. 
	\item What kind of methodologies are being worked upon in compmusic, What is the main philosophy
	\item The idea of open-acess and reproducibility
	\item what more?
\end{itemize}

\section{Computational melodic analysis of Indian art music: challenges and opportunities}
\label{sec:intro_challenges_oppurtunities}

%
%\begin{itemize}
%	\item in general music similaity, search and discovery. Different applications and context
%	\item melodic analysis->representation of tonal content often used->melodic representaiton->difficulaty in extraction
%	\item problems which are solved by the state of the art in melody extraction for different music types
%	\item for which music types melodic anlaysis make more sense and have been done successfully.
%	\item Symbolic domain research, a lot of work.
%	\item Key detection and cover song detection.
%\end{itemize}
%
%\begin{itemize}
%	\item Music similarity focus in MIR
%	\item application and context of music similarity
%	\item how that has given rise to search and discovery in different contexts
%	\item use cases/ applications / relevant problems within similarity and search and discovery
%\end{itemize}

\section{Scope and Broad Objectives}
\label{sec:intro_scope_context_relevance}

\section{Organization and Outline of the thesis}
\label{sec:intro_organization}
